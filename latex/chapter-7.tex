\documentclass{article}

\usepackage{hyperref}
\usepackage{tabu}
\usepackage{enumitem}
\begin{document}

\title{Table, list}

\maketitle





\subsection{Leitbild des ÖGD}\label{viv-id-https:003a:002f:002fconsortium:002egithub:002eio:002flt01:002fwebbuch:002fnew-netbuch:002exhtml:0023H8257368}



Die 89. \href{https://www.gmkonline.de/}{Gesundheitsministerkonferenz (GMK)} betonte 2016 in ihrem Grundsatzbeschluss die unverzichtbare Rolle des ÖGD im Gesundheitswesen, ausgehend vom Gesundheitsschutz, der Prävention und Gesundheitsförderung bis hin zur Gesundheitsversorgung. Diese Rolle sollte dabei allen politischen Ebenen und den Akteuren der Selbstverwaltung im Gesundheitswesen stärker präsent werden. Auch stellte die GMK fest, dass die Bezeichnung des ÖGD als „dritte Säule des Gesundheitswesens” neben der ambulanten und der stationären Versorgung die aktuellen bevölkerungsmedizinischen Herausforderungen des ÖGD nicht annähernd umfassend darstellte.


Um die Perspektiven für den ÖGD neu zu bestimmen und den ÖGD zu stärken, regte die GMK an, die öffentliche Wahrnehmung des OGD durch ein modernes Leitbild zu verbessern und initiierte, dass alle Träger des ÖGD und für den ÖGD engagierte Verbände und Institutionen sich in einem Prozess einbrachten, um gemeinsam ein modernes Leitbild für den ÖGD zu entwickeln.


Um die Tätigkeit des ÖGD auch zukünftig effektiv und effizient gestalten zu können, wies die GMK 2017 die Arbeitsgemeinschaft der Obersten Landesgesundheitsbehörden (AOLG) an: Der weitere Erfahrungsaustausch der Länder und der kommunalen Träger des ÖGD soll über Beispiele guter Praxis stetig gefördert werden. Sie beauftragte zudem eine länderoffene Arbeitsgruppe ÖGD, einen von den Ländern vorbereiteten Leitbildentwurf redaktionell zu bearbeiten und einen konkreten Konsultations- und Transferprozess mit dem ÖGD und den Verbänden zu organisieren.









\begin{tabu} to \textwidth { |X|X|X|X| }
\hline



Esto es el head & Esto es el head & Esto es el head & Esto es el head
 \\


vv & vv & vv & vv
 \\


vv & vv & vv & vv
 \\
\hline

\end{tabu}




Insbesondere Vertreterinnen und Vertreter kommunaler Gesundheitsämter erstellten ein Arbeitspapier, das sie dem AOLG und der GMK im März 2018 vorlegten und diese dazu berieten. Nach Abschluss der 91. GMK im März 2018urde als einstimmiger Beschluss der Ministerinnen und Minister sowie der Senatorinnen und Senatoren für Gesundheit der Länder das „\href{https://www.akademie-oegw.de/die-akademie/leitbild-oegd.html}{Leitbild eines modernen Öffentlichen Gesundheitsdienstes - Zuständigkeiten. Ziele. Zukunft.}” vorgestellt. Dieses formuliert folgende Kernaussagen:


\begin{tabu} to \textwidth { |X|X|X|X| }
\hline



\# & Make & Model & Year
 \\


1 & Honda & Accord & 2009
 \\


2 & Toyota & Camry & 2012
 \\


3 & Hyundai & Elantra & 2010
 \\
\hline

\end{tabu}




Der Öffentliche Gesundheitsdienst

\begin{enumerate}
\item Hat die öffentliche Verantwortung für die Gesundheit der Bevölkerung


\item Ist integraler Baustein des modernen Sozialstaats


\item Ist bürgernah und eingebunden in kommunale Strukturen


\item Orientiert sich an lokalen und globalen Herausforderungen


\item Ist gemeinwohlorientiert, ohne kommerzielle Interessen


\item Hat als Kernaufgaben Gesundheitsschutz, Gesundheitsförderung, Beratung und Information sowie Steuerung und Koordination


\item Nimmt hoheitliche Aufgaben wahr und arbeitet sozialkompensatorisch, planerisch und gestalterisch, um gesundheitliche Chancengleichheit und bestmögliche Gesundheit für alle zu ermöglichen (Public Health)


\item Basiert auf medizinischen, insbesondere fachärztlichen, und sozial- sowie gesundheitswissenschaftlichen Qualifikationen


\item Arbeitet wissenschaftsbasiert und vernetzt


\item Ist ethisch reflektiert in Respekt vor der Würde des einzelnen Menschen.


\end{enumerate}

\subsection{Der ÖGD in Zahlen}\label{viv-id-https:003a:002f:002fconsortium:002egithub:002eio:002flt01:002fwebbuch:002fnew-netbuch:002exhtml:0023H837015}



Der Föderalismus ist das staatliche Organisationsprinzip der Bundesrepublik Deutschland, einem Bundesstaat mit sechzehn Ländern als Gliedstaaten. Die Ausübung der Staatsgewalt ist durch das Grundgesetz zwischen Bund und Ländern aufgeteilt.


\begin{tabu} to \textwidth { |X|X|X| }
\hline



thead & thead & thead
 \\


this is table & this is table & this is table
 \\


this is table & this is table & this is table
 \\


this is table & this is table & this is table
 \\
\multicolumn{3}{c}{

this are merge 3 cells
} \\



 & \multicolumn{2}{c}{

merge 2 cells
} \\


this is table & this is table & this is table
 \\
\hline

\end{tabu}

Die Verwaltungsstrukturen und Zuständigkeiten der sechzehn Bundesländer sind nur zum Teil vergleichbar. Dreizehn der Bundesländer sind sogenannte Flächenländer, die drei Bundesländer Berlin, Bremen, Hamburg sind sogenannte Stadtstaaten. Die Flächenländer Bayern, Hessen und Nordrhein-Westfalen sind in Regierungsbezirke, Baden-Württemberg in Regionen aufgeteilt.


In den dreizehn Flächenländern sind die kleineren Gemeinden in Landkreisen oder ähnlichen Gemeindeverbänden zusammengefasst. Kleine und kleinste.


\begin{tabu} to \textwidth { |X|X|X|X| }
\hline



\emph{\textbf{Bundesland }}\emph{(link zur Quelle)} & \emph{\textbf{Anzahl Gesund-}}

\emph{\textbf{heitsämter}} & \emph{\textbf{Gesund-}}

\emph{\textbf{heitsämter pro 1 Mio. Einwohner}} & \emph{\textbf{Einwoh-}}

\emph{\textbf{ner pro Gesund-}}

\emph{\textbf{heitsamt}}
 \\


\href{https://www.gesundheitsamt-bw.de/lga/DE/Startseite/OEGD_BW/Gesundheitsaemter/Seiten/default.aspx}{Baden-}

\href{https://www.gesundheitsamt-bw.de/lga/DE/Startseite/OEGD_BW/Gesundheitsaemter/Seiten/default.aspx}{Württemberg} & 35 & 3,2 & 314.955
 \\


\href{https://www.lgl.bayern.de/gesundheit/gesundheitsaemter.htm}{Bayern} & 76 & 5,8 & 171.016
 \\


\href{https://service.berlin.de/standorte/gesundheitsaemter/}{Berlin} & 12 & 3,3 & 301.125
 \\


\href{https://service.brandenburg.de/lis/detail.php/118431}{Brandenburg} & 19 & 7,6 & 131.792
 \\


\href{https://www.gesundheitsamt.bremen.de/}{Bremen} & 1 & 1,5 & 681.032
 \\


\href{https://hamburg.offenesamt.de/gesundheitsamt/}{Hamburg} & 7 & 3,8 & 261.512
 \\


\href{https://www.laekh.de/buerger-patienten/wichtige-adressen/gesundheitsaemter}{Hessen} & 24 & 3,8 & 260.136
 \\


\href{https://www.lagus.mv-regierung.de/Gesundheit/Gesundheits%C3%A4mter/}{Mecklenburg-}

\href{https://www.lagus.mv-regierung.de/Gesundheit/Gesundheits%C3%A4mter/}{Vorpommern} & 8 & 5 & 201.390
 \\


\href{http://www.hygieneinspektoren-nds.de/gesundheitsaemter_nds}{Niedersachsen} & 45 & 5,7 & 176.951
 \\


\href{https://www.lzg.nrw.de/service/links/gesundheitsaemter_nrw/index.html}{Nordrhein-}

\href{https://www.lzg.nrw.de/service/links/gesundheitsaemter_nrw/index.html}{Westfalen} & 60 & 3,3 & 298.536
 \\


\href{https://paartherapeut-finden.de/gesundheitsaemter/189}{Rheinland-}

\href{https://paartherapeut-finden.de/gesundheitsaemter/189}{Pfalz} & 25 & 6,1 & 162.947
 \\


\href{https://www.saarland.de/4080.htm}{Saarland} & 6 & 6 & 165.698
 \\


\href{https://consortium.github.io/https://www.gesunde.sachsen.de/6849.html}{Sachsen} & 13 & 3,2 & 313.947
 \\


\href{https://ms.sachsen-anhalt.de/themen/gesundheit/daten-zur-gesundheit/badegewaesser/kontakte/gesundheitsaemter/}{Sachsen-}

\href{https://ms.sachsen-anhalt.de/themen/gesundheit/daten-zur-gesundheit/badegewaesser/kontakte/gesundheitsaemter/}{Anhalt} & 14 & 6,3 & 158.792
 \\


\href{https://www.schleswig-holstein.de/DE/Fachinhalte/G/gesundheits_dienste/Downloads/OeffentlicherGesundheitsdienst/listeGesAemter.html}{Schleswig-}

\href{https://www.schleswig-holstein.de/DE/Fachinhalte/G/gesundheits_dienste/Downloads/OeffentlicherGesundheitsdienst/listeGesAemter.html}{Holstein} & 15 & 5,2 & 192.655
 \\


\href{https://www.schleswig-holstein.de/DE/Fachinhalte/G/gesundheits_dienste/Downloads/OeffentlicherGesundheitsdienst/listeGesAemter.html}{Schleswig-}

\href{https://www.schleswig-holstein.de/DE/Fachinhalte/G/gesundheits_dienste/Downloads/OeffentlicherGesundheitsdienst/listeGesAemter.html}{Holstein} & 15 & 5,2 & 192.655
 \\


\href{https://www.schleswig-holstein.de/DE/Fachinhalte/G/gesundheits_dienste/Downloads/OeffentlicherGesundheitsdienst/listeGesAemter.html}{Schleswig-}

\href{https://www.schleswig-holstein.de/DE/Fachinhalte/G/gesundheits_dienste/Downloads/OeffentlicherGesundheitsdienst/listeGesAemter.html}{Holstein} & 15 & 5,2 & 192.655
 \\


\href{https://www.schleswig-holstein.de/DE/Fachinhalte/G/gesundheits_dienste/Downloads/OeffentlicherGesundheitsdienst/listeGesAemter.html}{Schleswig-}

\href{https://www.schleswig-holstein.de/DE/Fachinhalte/G/gesundheits_dienste/Downloads/OeffentlicherGesundheitsdienst/listeGesAemter.html}{Holstein} & 15 & 5,2 & 192.655
 \\


\href{https://www.schleswig-holstein.de/DE/Fachinhalte/G/gesundheits_dienste/Downloads/OeffentlicherGesundheitsdienst/listeGesAemter.html}{Schleswig-}

\href{https://www.schleswig-holstein.de/DE/Fachinhalte/G/gesundheits_dienste/Downloads/OeffentlicherGesundheitsdienst/listeGesAemter.html}{Holstein} & 15 & 5,2 & 192.655
 \\


\href{https://www.thueringen.de/th3/tlvwa/gesundheit/oeffentlicher_gesundheitsdienst/aemter/}{Thüringen} & 23 & 10,7 & 93.531
 \\
\hline

\end{tabu}

Tabelle 1. Anzahl der Gesundheitsämter in Bundesländern und Anzahl

\begin{quote}



\emph{“ein Zustand des vollkommenen körperlichen, geistigen und sozialen Wohlbefindens und nicht allein das Fehlen von Krankheit oder Gebrechen." (Health is a state of complete physical, mental and social well-being and not merely the absence of disease or infirmity). }


\end{quote}







\begin{tabu} to \textwidth { |X|X|X| }
\hline



This a table from Google & Gdocs & 


 \\


\emph{\textbf{Gesund-}}


\emph{\textbf{heitsämter pro 1 Mio. Einwohner}} & \emph{\textbf{Anzahl Gesund-}}


\emph{\textbf{heitsämter}} & \emph{\textbf{Einwoh-}}


\emph{\textbf{ner pro Gesund-}}


\emph{\textbf{heitsamt}}
 \\


35 & 3,2 & 314.955
 \\
\hline

\end{tabu}





\begin{quote}



\subsection{Rapid Collaborative Health Publishing info \href{https://github.com/TIBHannover/Rapid-Collaborative-Health-Publishing}{https://github.com/TIBHannover/Rapid-Collaborative-Health-Publishing}}\label{H1636129}


\begin{itemize}
\item openVirus \href{https://github.com/petermr/openVirus/blob/master/OVERVIEW.md}{https://github.com/petermr/openVirus/blob/master/OVERVIEW.md}


\item Hack signup \href{https://euvsvirus.org/}{https://euvsvirus.org/}


\end{itemize}

Installing ContentMine, also see READMe.md \href{https://github.com/petermr/openVirus/blob/master/INSTALLING.md}{https://github.com/petermr/openVirus/blob/master/INSTALLING.md}


\subsection{\textbf{Fidus - Git Update SW 13.4.2020}}\label{H3101768}


\begin{itemize}
\item \textbf{Objective:} automate the workflow from Fidus Writer to multi-format outputs via GitHub


\item \textbf{Priority:} we have the ‘contact tracing’ publication coming up soon which will be an extremely important publication (no doubt with many follow on publications and revisions) we need to be able to process this as fast as possible for use in Germany, Europe and internationally. 


\end{itemize}

For this reason solving problems in having Fidus Writer ‘publication package’ styling and internal workings is more important than full automation.


Below are three sections to help make a plan of action:

\begin{itemize}
\item \textbf{Workflow}


\item \textbf{Route options for outputs}


\item \textbf{Current ‘publication package’ issues}


\end{itemize}

\textbf{Action points: }when we have figured out a plan or action I will talk to people like: Vivliostyle, EElectric, Mikro Verlag and look for high quality pointers of start templates.




\subsubsection{\textbf{Options for Fidus Writer publication package in Workflow Plan}}\label{H9605788}



\subsubsection{\textbf{EPUB and then post process}}\label{H5047983}


\begin{itemize}
\item EPUB

\begin{itemize}
\item One embedded CSS book style from Fidus Writer (we hit one issue in that we do a number of things which will behave or not work in the variable page ebook but are needed in Web-buch and PDFs, these are: in page ToC, blank pages, citation positioning, etc. A possible solution is to have A. Several CSS files included in Fidus Writer Theme as in - print.css, ebook.css, web-buch.css. OR make several Fidus Writer themes?

\begin{itemize}
\item Vivliostyle reads - uncompressed or compressed EPUB

\begin{itemize}
\item Web-buch Vivliostyle in-browser CSS Book


\item Generate PDF in browser manually. Metadata and bookmarks can be added with CLI \href{https://github.com/Mute-Publishing/Archive-Documentation#add-bookmarks-and-basic-metadata}{https://github.com/Mute-Publishing/Archive-Documentation\#add-bookmarks-and-basic-metadata}

\begin{itemize}
\item Screen PDF - Save in repo /pdf


\item PoD PDF - Manually removed cover for PoD PDF, and make cover in Indesign, Scribus, or CSS Typesetting \href{https://www.pagedmedia.org/a-book-by-its-cover/}{https://www.pagedmedia.org/a-book-by-its-cover/}


\end{itemize}

\item Jekyll GitHub pages - Use XHTML files as is, or convert to GitHub MD. If we leave as XHTML then we need to find a Jekyll theme that will work with XHTML or HTML.  (Jekyll always a bit trick as need to put bits of info into different config files to get everything right)

\begin{itemize}
\item Jekyll could also read from the HTML export


\item If HTML or XHTML used then CSS path would need to be changed to use Jekyll CSS resources


\end{itemize}

\end{itemize}

\end{itemize}

\item We would want EPUB CSS to work in Fidus Writers other outputs: PDF and HTML. So users could preview or use books direct. 


\item We have to deal with getting correct metadata to each output


\item We need a source output: whatever we consider the most complete and interoperable format? JATS, EPUB, HTML, ALL?


\end{itemize}

\end{itemize}



\textbf{Header2}


\subsubsection{Header3}\label{H6081916}



\subsubsection{Header4}\label{H5874541}


\begin{itemize}
\item der Staat mit seiner Gesetzgebung und seinen öffentlichen Körperschaften


\item die Leistungsempfänger (Patientinnen und Patienten)


\item die Leistungserbringer (z.B. Ärzte/innen, Apotheker/innen, Therapeuten/innen und Pflegepersonal 


\item 
\begin{itemize}
\item der Staat mit seiner Gesetzgebung und seinen öffentlichen Körperschaften


\item die Leistungsempfänger (Patientinnen und Patienten)


\item die Leistungserbringer (z.B. Ärzte/innen, Apotheker/innen, Therapeuten/innen und Pflegepersonal u.a.)


\item die Leistungsfinanzierer (Kranken-, Pflege- oder Rentenversicherungen)


\item Interessenverbände, Interessenvertretungen.u.a.)


\end{itemize}

\item die Leistungsfinanzierer (Kranken-, Pflege- oder Rentenversicherungen)


\item Interessenverbände, Interessenvertretungen.


\end{itemize}

\end{quote}


\subsubsection{\textbf{Questions and issues}}\label{H6787224}



\subsubsection{\textbf{General:}}\label{H337015}



Copy list from Gdocs

\begin{enumerate}
\item Fidus Writer – Real-time collaborative WYSIWYG editor


\item GitHub infrastructure – versioning, crypto IDs, API, web server, CI/CD, etc


\item Scalable – modern cloud infrastructure from Endocode meaning instant deployment, low costs, etc


\item Multi-format publication ready outputs (PRO) – 100\% standards validation and distribution channels ready


\item Multiverse interoperability – publications ready for use in format domains: R, RDF, XML, etc.


\item Open Science Service Ready – publication sources on GitHub makes content ready for processing by content services (e.g., smart objects) and data science services (e.g., PiDs), with metering.


\item Multilingual R/L, L/R translation – CI driven translation on GitLab via Crowdin or Weblate (courtesy Modern Publishing, Hamburg Open Science)


\item Real-time validation – validate against defined targets to ensure content 100\% usable, with interactive GUI - phase two Transpect pipeline feature.


\item Service metering – provide metrics and monitoring to support a sustainable economy by providers of services and skilled labor.


\item Interface Moodle and LMSs – enable bi-direction LMS support

\begin{itemize}
\item Rapid Collaborative Health Publishing info \href{https://github.com/TIBHannover/Rapid-Collaborative-Health-Publishing}{https://github.com/TIBHannover/Rapid-Collaborative-Health-Publishing}


\item openVirus \href{https://github.com/petermr/openVirus/blob/master/OVERVIEW.md}{https://github.com/petermr/openVirus/blob/master/OVERVIEW.md}


\item Hack signup \href{https://euvsvirus.org/}{https://euvsvirus.org/}


\item Installing ContentMine, also see READMe.md \href{https://github.com/petermr/openVirus/blob/master/INSTALLING.md}{https://github.com/petermr/openVirus/blob/master/INSTALLING.md}


\end{itemize}

\item Metadata – extensive, publication, assets, LOD and Wikidata, social media, and usage and sharing, PIDs


\item Contributors and roles – rewards, attribution


\item Community publication features – enable awards, credit and acknowledgements with pathways and on-ramps to contribute


\item Designed for next generation scholarly publishing – standards, citations, etc


\end{enumerate}




Copy from GOOGLE >>>>>


\subsubsection{\textbf{Options for Fidus Writer publication package in Workflow Plan}}\label{H428481}



\subsubsection{\textbf{EPUB and then post process}}\label{H5347513}



Nested list copy it fro Gdocs

\begin{itemize}
\item EPUB

\begin{itemize}
\item One embedded CSS book style from Fidus Writer (we hit one issue in that we do a number of things which will behave or not work in the variable page ebook but are needed in Web-buch and PDFs, these are: in page ToC, blank pages, citation positioning, etc. A possible solution is to have A. Several CSS files included in Fidus Writer Theme as in - print.css, ebook.css, web-buch.css. OR make several Fidus Writer themes?

\begin{itemize}
\item Vivliostyle reads - uncompressed or compressed EPUB

\begin{itemize}
\item Web-buch Vivliostyle in-browser CSS Book


\item Generate PDF in browser manually. Metadata and bookmarks can be added with CLI \href{https://github.com/Mute-Publishing/Archive-Documentation#add-bookmarks-and-basic-metadata}{https://github.com/Mute-Publishing/Archive-Documentation\#add-bookmarks-and-basic-metadata}

\begin{itemize}
\item Screen PDF - Save in repo /pdf


\item PoD PDF - Manually removed cover for PoD PDF, and make cover in Indesign, Scribus, or CSS Typesetting \href{https://www.pagedmedia.org/a-book-by-its-cover/}{https://www.pagedmedia.org/a-book-by-its-cover/}


\end{itemize}

\item Jekyll GitHub pages - Use XHTML files as is, or convert to GitHub MD. If we leave as XHTML then we need to find a Jekyll theme that will work with XHTML or HTML.  (Jekyll always a bit trick as need to put bits of info into different config files to get everything right)

\begin{itemize}
\item Jekyll could also read from the HTML export


\item If HTML or XHTML used then CSS path would need to be changed to use Jekyll CSS resources


\end{itemize}

\end{itemize}

\end{itemize}

\item We would want EPUB CSS to work in Fidus Writers other outputs: PDF and HTML. So users could preview or use books direct. 


\item We have to deal with getting correct metadata to each output


\item We need a source output: whatever we consider the most complete and interoperable format? JATS, EPUB, HTML, ALL?


\end{itemize}

\end{itemize}




--------


\subsubsection{\textbf{Multi-format publication package issues}}\label{H7608727}


\begin{enumerate}
\item We have a series of problems with headers at the start of Chapters. Document title is going in, if we add Chapter description, this also gets outputted. Sometimes editors are adding in title again (but this is user error). 


\item The outputted ToC page appears in ebook and HTML. In ebook its showing up at end of book, also Cover is as well.


\item Can we control ToC header level in EPUB from fidus writer.


\item Out putted HTML packages have document.css and book style.css.

\begin{enumerate}
\item Where does document.css come from? ===>>> this one should be a.


\item Is the CSS file of the style from the book style or document style. If document style used  ===>>> this one should be b.


\item 


\end{enumerate}

\item Is it OK to add image into the style file upload for things specific to the style. Do ebooks like this?


\item 


\item .WOFF font types used. Can we use other font types? If not how do we make .WOFF font types?


\item Is Fidus Writer using Vivliostyle when it generates PDFs in Chrome, if so what version - is it the same as the Vivlio viewer we are using.


\item We’d like to add our styles into Fidus, is there a recommended base CSS to start from?


\item If we want to add several CSS files and CSS address paths into Fidus Writer do we think we could do that.


\item We have had problems with current MD not writing figure caption into MD


\item Also current MD is not outputting citations


\item We have problems getting the figure numbers to count sequentially in a book and not restart each chapter


\end{enumerate}

---


\subsubsection{\textbf{Key findings}}\label{H2686412}


\begin{itemize}
\item We use MoSCoW to prioritise if needed. (Must have, Should have, Could have, and Won't have)


\item Points 1 to 10 are listed here and then at 10. It refers back to full notes as it started to become the case that all the items apply.


\end{itemize}
\begin{enumerate}
\item Need to be able to upload file types to specific places in ‘template repository’, e.g., .epub to /ebook.


\item That we need to merge FW files with ‘ADA template repository’.


\item Is there a recommendation about what FW book theme we should use as a starting theme? 


\item Design ADA Template repository @simon to lead. NB consult Axel Dürkop on multilingual issues.


\item Resolve issues in ‘\href{https://docs.google.com/document/d/1ExUG-KRUlHns2MsvxKGFdBY8t1Exe2Nl56E_yh9IWi4/edit#heading=h.88hftk6fyv22}{Fidus Writer content}’ section:

\begin{enumerate}
\item Do EPUB 3.0 ebook devices read ’ CSS so that we can write simple ebook theme. @raquel to resolve and test.


\item Scholarly-markdown conversion, how should we do this and will we use HTML or EPUB source.


\item Can the HTML output be organised into nested folders for content types: fonts, styles, images, js, etc. This is not a priority: COULD.


\item Sources: confirm what will be used. HTML There is a book version of JATS


\end{enumerate}

\item General issues:

\begin{enumerate}
\item A general issue is that we want sections (AKA chapters) to be in different orders depending on outputs. Some type of section in the book area for output order per format for sections should not be included in specific output. @simon will make an example grid for the current ‘crisis management’ books as an example.


\item Adding tracking codes and what type of tracking codes, Matomo, Google? This is not a priority: COULD.


\item How do we make sure outputs of repository are in synch across different outputs


\item Can we have some sort of ID, hash, version number to know which version a file came from. Note on GitHub we will use releases for publication versions.


\end{enumerate}

\item Re: 1. a. Need to make a schematic of FW internals as not completely sure about what is there, how related and their function/role. @simon


\end{enumerate}

----

\begin{enumerate}
\item Getting to understand the internals of FW: 

\begin{enumerate}
\item Documents, books, images, bibliography, document templates (page type?), document styles, book styles, export templates, citation styles, language settings, book data: order of chapters, chapter descriptions, cover, metadata. Styling editing view of document - I think it’s done in document template but not sure?


\item Book theme - one CSS file, fonts, and images. Can we use SVG files? Can we upload, can they be used in our different outputs? NB vector PNG can be used. Again question is do these work on all outputs?


\item Document template - field types. 


\end{enumerate}

\item Web-buch > we’re also meaning screen PDF and PoD PDF (some manual parts take place with PoD PDF but put these aside for the moment. Book order: covers, front matter, text, back matter.

\begin{enumerate}
\item Web-buch: check and confirm browser compatibility, on mobile too, e.g., iPad?


\item Web-buch: we have issues that different browsers give different content lengths, so some rendering take up more pages. This had knock on effects: list TBC.






\end{enumerate}

\end{enumerate}
\begin{enumerate}
\item 
\begin{enumerate}
\item 
\begin{enumerate}[start=3]
\item Front cover (fc) and back covers (bc): Currently added manually and via CSS as background images. Fc easy because this page 1, bc not easy as we dont know number until book is rendered. Our convention is to use standard file names: cover.png and back.cover.png (PNG as then we can use vector based images)? I would imagine that we add these to the book area for upload. NB: problem controlling pages at end of book in Vivliostyle, adding blanks, adding page background image. How to manage this from FW.


\item Front matter needs special styling and ToC: (we think solution is by using different document templates and give them IDs that can be used by CSS, then it would be possible to make title page and front matter sections and give them styles and have them not appear in ToC. If we had different document templates, one for ‘front matter’, one for ‘ToC’ then this might be the solution?

\begin{enumerate}
\item Title page - needs own style, Document title not to appear. Note when no title used we have problems in FW document management interface as managing lots of documents called ‘Untitled’ doesn’t work.


\item Whole of front matter need own page style: no folios, no page numbers, page breaks, typographic style.


\item Page breaks as a FW editor markup: maybe being able to add page breaks in the editor in combination to being able to add a ID for CSS to document templates might be useful for dealing with front matter.


\item ToC: 

\begin{enumerate}
\item In page ToC: recto, currently part manual, linked by manually giving chapters and ID, adding ID to ToC to give an anchor link and write in chapters name. Page numbers are made automatically. How can we automate the whole ToC process?


\item Menu ToC: OK.


\end{enumerate}

\end{enumerate}

\item Text: Chapter starts issue - chapter name recto, blank page verso opposite, text to start on the following verso page. Problem with blank verso, currently done with CSS and problem is that text want to fall before chapters starts.


\end{enumerate}

\end{enumerate}

\end{enumerate}

\subsubsection{\textbf{Questions for Vivlio}}\label{H1976743}


\begin{itemize}
\item this is a list 

this is a nested list


\end{itemize}
\begin{enumerate}
\item Tips for solving content overrun issues in different browsers


\item Problem controlling pages at end of book in Vivliostyle, can’t insert blanks and image background.



i want to make a list



Low hanging fruit:


\end{enumerate}






\begin{itemize}
\item How can current book be internationalise? 




\item Give two or three project to funders: crisis management book , book on german public health service


\item How to use open-source technology that has not been used before


\end{itemize}

i ma righting normal

\end{document}
